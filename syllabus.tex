% Created 2025-05-27 Tue 15:03
% Intended LaTeX compiler: pdflatex
\documentclass[11pt]{article}
\usepackage[utf8]{inputenc}
\usepackage[T1]{fontenc}
\usepackage{graphicx}
\usepackage{longtable}
\usepackage{wrapfig}
\usepackage{rotating}
\usepackage[normalem]{ulem}
\usepackage{amsmath}
\usepackage{amssymb}
\usepackage{capt-of}
\usepackage{hyperref}
\author{fcalado}
\date{\today}
\title{}
\hypersetup{
 pdfauthor={fcalado},
 pdftitle={},
 pdfkeywords={},
 pdfsubject={},
 pdfcreator={Emacs 29.3 (Org mode 9.6.15)}, 
 pdflang={English}}
\begin{document}

\tableofcontents

\section{syllabus}
\label{sec:org48cc3ba}
\subsection{course information}
\label{sec:org358bd5d}
Summer 2025

606 Pratt Manhattan Campus 

M/Th 10:00 AM - 1:50 PM

5/27/2025 - 7/8/2025

Credits: 3

Prerequisites or other restrictions: INFO 664

\subsection{instructor contact information}
\label{sec:org2583913}
Filipa Calado, PhD (she/her)

Office Location: Pratt Manhattan Campus, room 602

Office Hours: by appointment (summer only)

Phone: 718-687-5194

Email: fcalado@pratt.edu

Zoom: \url{https://pratt.zoom.us/my/fcalado}

Website: \url{https://filipacalado.com} 

\subsection{course description}
\label{sec:org7c1a7fc}
This course offers a practical introduction to building bots in Python
alongside a critical examination of algorithmic bias. Students will
learn core programming skills like data collection and API interaction
in order to create web crawlers and social media bots. In parallel,
students will explore the ethical consequences of automated systems on
social platforms, such as the amplification of misinformation and bias
that perpetuate social inequalities and discrimination. This course
will equip students with the critical perspectives and technical
skills to analyze automated systems in a world increasingly shaped by
AI and algorithmic decision-making. This course is intended for
students interested in both technical development and the social
impacts of automation.

\subsection{student learning outcomes:}
\label{sec:org93e287f}
Upon successful completion of this course, students will be able to:
\begin{itemize}
\item develop programming literacies for working with popular software in
Python for web crawling and API usage.
\item gain hands-on experience in computational data collection and
parsing.
\item identify and explain how choices in bot design, such as source
selection, keyword filtering, and data processing, can introduce
and/or amplify biases.
\item critically assess the societal impact of bots, especially those used
in social media, evaluating how bots might influence information
visibility and public discourse.
\end{itemize}

\subsection{course format}
\label{sec:org0fb70fe}
This course will be held in hybrid format, with the first four weeks
of meetings (May 29 - June 23) in person, and the last two weeks of
meetings (June 26 - July 7) on zoom.

Our sessions will be split evently between lecture, individual
practice, and group work. This contrasts with the prerequisite course,
INFO 664, which focused more heavily on lecture.

For each meeting, one 30-minute break will take place from
approximately 12:00pm-12:30pm.

\subsection{course materials}
\label{sec:org2356884}
Having a personal laptop (not a tablet) where you can install software
is essential for this class.

All assignments and readings will be provided electronically and
hosted on github at \url{https:gofilipa.github.io/how2bot}.

\subsection{class communication}
\label{sec:orgc735b7f}
The instructor will contact you via your pratt email (linked to
Canvas). If you don't check that email frequently, please remember to
do so for this class or set up mail forwarding.

The best avenue for contacting the instructor is via email, at
fcalado@pratt.edu. Response time should be within 2 business days,
otherwise feel free to follow up. 

\subsection{course schedule}
\label{sec:org1f3d169}
\subsubsection{unit 1 web crawling bots - 2 weeks}
\label{sec:org35f4d70}
\begin{enumerate}
\item May 29, session 1: intro to Python \& web scraping with bs4
\label{sec:org96d8473}

homework: find 2 scrapable sites
\begin{itemize}
\item find 2 websites to scrape. Make sure if they are scrapable with bs4.
Why are you interested in this data? What could you do with it?
\end{itemize}

\item June 2, session 2: scrapy \& the scrapy shell
\label{sec:org5bf504e}

homework: reading response \emph{Compost Engineers} chapters 1 \& 2
\begin{itemize}
\item Joana Varon and Lucía Egaña Rojas. Chapters 1 \& 2 from \emph{Compost
Engineers and Sus Saberes Lentos: A Manifest for Regenerative
Technologies}. Coding Rights, 2024,
\url{https://codingrights.org/docs/compost\_engineers.pdf}.
\item Prompt: Pick an idea from the reading that interests you (either
because you agree with it, disagree with it, or are otherwise
provoked by it) and explain why. 1 page.
\end{itemize}

\item June 5, session 3: blockers \& XHR
\label{sec:orga08632a}

homework: \emph{Compost Engineers} chapters 3 \& 4
\begin{itemize}
\item Joana Varon and Lucía Egaña Rojas. Chapters 3 \& 4 from \emph{Compost
Engineers and Sus Saberes Lentos: A Manifest for Regenerative
Technologies}. Coding Rights, 2024,
\url{https://codingrights.org/docs/compost\_engineers.pdf}.
\item Prompt: From the authors' proposals, what do you find useful or
surprising, and what do you have doubts about? 1 page.
\end{itemize}

\item June 9, session 4: selenium
\label{sec:org85431e1}

assignment: web scraping
\begin{itemize}
\item Using either scrapy or selenium, scrape some data from a website
that you couldn't scrape before.
\end{itemize}
\end{enumerate}

\subsubsection{unit 2 chat bots - 1.5 weeks}
\label{sec:org1b15ebc}
\begin{enumerate}
\item June 12, session 5: spaCy for processing text
\label{sec:orgf5b4ba2}

homework: ACLU tech \& privacy analysis write-up
\begin{itemize}
\item Choose a recent topic from this page; write up analysis of what is
going on, and your opinion on the issue. How does the issue handle
privacy rights and ethical uses of data?
\url{https://www.aclu.org/press-releases?issue=privacy-technology}
\end{itemize}

\item June 16, session 6: spacy continued, intro to transformers
\label{sec:org5f3bdf9}
homework: run a task on your own data

\item June 19, session 7: transformers continued
\label{sec:org00d8e05}

assignment: dataset proposal
\begin{itemize}
\item What is the dataset you'd like to create for your final project?
Where would you get the data, and how would you transform it? You
can consider tools from this class (like text generation, named
entity recognition, pattern matching), or you can consider other
possibilities for transforming your data. 1 page, double spaced.
\end{itemize}
\end{enumerate}

\subsubsection{unit 3 social media bots - 1.5 weeks}
\label{sec:org1f8ac74}
\begin{enumerate}
\item June 23, session 8: twitter bots
\label{sec:org15265f2}

homework: make a plan for actions steps by next class

\item (online) June 26, session 9: group projects
\label{sec:orge954c33}

homework: work on projects

\item (online) June 30, session 10: group projects continued
\label{sec:org4c91f5e}

homework: project proposal
\end{enumerate}

\subsubsection{unit 4 project workshops \& presentations - 1 week}
\label{sec:orgff9d547}
\begin{enumerate}
\item (online) July 3, final projects
\label{sec:org3b71eba}

homework: work on projects

\item (online) July 7, final project presentations
\label{sec:org644bc4e}
\end{enumerate}

\subsection{assignments}
\label{sec:orgdf94890}
\subsubsection{participation (30\%)}
\label{sec:org599c846}
\begin{itemize}
\item Includes in-class engagement, and completing and sharing homework
\end{itemize}
\subsubsection{unit assignments (30\%)}
\label{sec:org7d6ca43}
\begin{itemize}
\item Average score of 3 assignments at the end of units 1-3
\end{itemize}
\subsubsection{final project: some bot! (40\%)}
\label{sec:org67e1a46}
\begin{itemize}
\item A final project that takes some data from web scraping or APIs, and
uses it as the content for a bot.
\item bot to be automated and published on github.
\end{itemize}

\subsection{recommended readings}
\label{sec:org0e7001c}
\subsubsection{on data gathering and web scraping}
\label{sec:org24293ca}
\begin{itemize}
\item Dodge, Jesse, et al. “Documenting Large Webtext Corpora: A Case
Study on the Colossal Clean Crawled Corpus.” Proceedings of the 2021
Conference on Empirical Methods in Natural Language Processing,
edited by Marie-Francine Moens et al., Association for Computational
Linguistics, 2021, pp. 1286–305. ACLWeb,
\url{https://doi.org/10.18653/v1/2021.emnlp-main.98}.
\item Jo, Eun Seo, and Timnit Gebru. “Lessons from Archives: Strategies
for Collecting Sociocultural Data in Machine Learning.” Proceedings
of the 2020 Conference on Fairness, Accountability, and
Transparency, Association for Computing Machinery, 2020, pp. 306–16.
ACM Digital Library, \url{https://doi.org/10.1145/3351095.3372829}.
\item Chan, Anita Say. Predatory Data: Eugenics in Big Tech and Our Fight
for an Independent Future. University of California Press, 2025.
library.oapen.org, \url{https://doi.org/10.1525/luminos.215}.
\item Métraux, Julia. “Eugenics Isn’t Dead—It’s Thriving in Tech.” Mother
Jones,
\url{https://www.motherjones.com/politics/2025/01/eugenics-isnt-dead-its-thriving-in-tech/}.
Accessed 14 Feb. 2025.
\end{itemize}

\subsubsection{on machine learning}
\label{sec:orgb559504}
\begin{itemize}
\item Alammar, Jay. The Illustrated BERT, ELMo, and Co. (How NLP Cracked
Transfer Learning). \url{https://jalammar.github.io/illustrated-bert/}.
Accessed 14 Apr. 2025.
\item Alammar, Jay. The Illustrated DeepSeek-R1. 10 Feb. 2025,
\url{https://newsletter.languagemodels.co/p/the-illustrated-deepseek-r1}.
\end{itemize}

\subsubsection{case studies of algorithmic bias \& audits}
\label{sec:org780eecf}
\begin{itemize}
\item Hada, Rishav, et al. “Akal Badi Ya Bias: An Exploratory Study of
Gender Bias in Hindi Language Technology.” The 2024 ACM Conference
on Fairness, Accountability, and Transparency, ACM, 2024, pp.
1926–39. DOI.org (Crossref),
\url{https://doi.org/10.1145/3630106.3659017}.
\item Gajjala, Radhika, et al. “Get the Hammer out! Breaking Computational
Tools for Feminist, Intersectional ‘Small Data’ Research.” Journal
of Digital Social Research, vol. 6, no. 2, 2, May 2024, pp. 9–26.
jdsr.se, \url{https://doi.org/10.33621/jdsr.v6i2.193}.
\item Tang, Ningjing, et al. “AI Failure Cards: Understanding and
Supporting Grassroots Efforts to Mitigate AI Failures in Homeless
Services.” The 2024 ACM Conference on Fairness, Accountability, and
Transparency, ACM, 2024, pp. 713–32. DOI.org (Crossref),
\url{https://doi.org/10.1145/3630106.3658935}.
\item Groves, Lara, et al. “Auditing Work: Exploring the New York City
Algorithmic Bias Audit Regime.” The 2024 ACM Conference on Fairness,
Accountability, and Transparency, ACM, 2024, pp. 1107–20. DOI.org
(Crossref), \url{https://doi.org/10.1145/3630106.3658959}.
\item Costanza-Chock, Sasha, et al. “Who Audits the Auditors?
Recommendations from a Field Scan of the Algorithmic Auditing
Ecosystem.” Proceedings of the 2022 ACM Conference on Fairness,
Accountability, and Transparency, Association for Computing
Machinery, 2022, pp. 1571–83. ACM Digital Library,
\url{https://doi.org/10.1145/3531146.3533213}.
\end{itemize}
\end{document}
